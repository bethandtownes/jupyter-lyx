\documentclass[english]{article}
\usepackage[T1]{fontenc}
\usepackage[utf8]{inputenc}
\usepackage{CJK}
\usepackage{babel}

% It is impossible to get the document language because the document could start with a command,
% whitespace, and English word or whatever and the rest is in Japanese. Checking for the first CJK
% environment is no solution because the document could be English and contain only some Japanese.

\begin{document}
 
\begin{CJK}{EUC-JP}{}%
Japanese \end{CJK} \begin{CJK}{UTF8}{}Chinese traditional\end{CJK} \begin{CJK}{EUC-JP}{}
Japanese \end{CJK}
hello
\begin{CJK}{GB}{}%
Chinese simplified \end{CJK}\begin{CJK}{EUC-JP}{hei} Japanese
(this CJK environment will be put in ERT because LyX supports only one CJK
font mapping per document, this environment uses the \texttt{hei} mapping,
and the first mapping is empty.\end{CJK} \begin{CJK}{GB}{}Chinese simplified
\end{CJK}

% The following Bg5 encoded text cannot be processed as is by latex.
% It needs to be preprocessed by bg5conv or bg5latex needs to be used.
% See the documentation of CJK.sty for details.

\begin{CJK}{Bg5}{}
Big5 �幩����
\end{CJK}

\inputencoding{latin9}%
\selectlanguage{english}%
English

% The following SJIS encoded text cannot be processed as is by latex.
% It needs to be preprocessed by sjisconv or sjislatex needs to be used.
% See the documentation of CJK.sty for details.

\begin{CJK}{SJIS}{}
Shift\_JIS ���{��̕���
\end{CJK}

hello

\begin{CJK}{JIS}{}
JIS-code $BF|K\8l$NJ8>O(B
\end{CJK}

\begin{CJK}{KS}{}%
Korean
\end{CJK}

\end{document}
